Py\-Earth is a lightweight Python package to support various Earth science tasks. It is designed to be a general purpose library as it is inspired by the popular I\-D\-L Coyote libarary (\href{http://www.idlcoyote.com/}{\tt http\-://www.\-idlcoyote.\-com/}).

Some of the code structure is inspired by the Arc\-G\-I\-S toolbox.

\section*{Content}

Py\-Earth mainly provides many general purpose funcations to support other libaries. These functions are classified into several categories\-:
\begin{DoxyEnumerate}
\item G\-I\-S\-: This component provides major spatial dataset operations.
\item Toolbox\-: This component provides many fuctions for data, date, math, etc.
\item Visual\-: This component provides plotting function for time series, scatter, etc.
\item System\-: This component provides system wide operations.
\end{DoxyEnumerate}

You can either call these functions through this package, or you can modify them for your own applications.

\section*{Acknowledgement}

This research was supported as part of the Next Generation Ecosystem Experiments-\/\-Tropics, funded by the U.\-S. Department of Energy, Office of Science, Office of Biological and Environmental Research at Pacific Northwest National Laboratory. The study was also partly supported by U.\-S. Department of Energy Office of Science Biological and Environmental Research through the Earth and Environmental System Modeling program as part of the Energy Exascale Earth System Model (E3\-S\-M) project.

\section*{Install}


\begin{DoxyEnumerate}
\item pip install pyearth
\end{DoxyEnumerate}

\section*{Contact}

Please contact Chang Liao (\href{mailto:changliao.climate@gmail.com}{\tt changliao.\-climate@gmail.\-com}) if you have any questions. 